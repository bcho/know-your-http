\documentclass[final,table]{beamer}

\usecolortheme{dove}

\usepackage{graphicx}
\usepackage{ctex}
\usepackage{fontspec, xunicode}
\usepackage[orientation=portrait,size=a1]{beamerposter}

\usepackage{scrextend}
\changefontsizes[20pt]{16pt}

\definecolor{light-gray}{gray}{0.95}

\setlength{\tabcolsep}{14pt}
\renewcommand{\arraystretch}{1.1}
\renewcommand{\baselinestretch}{1.25}
\setlength{\parindent}{0em}

\usepackage{cclicenses}

\beamertemplatenavigationsymbolsempty

\setbeamertemplate{footline}
{%
\begin{beamercolorbox}{}
    \hspace{25pt} \normalsize Know Your HTTP \input{./version.tex} \copyright \hspace{0in} The Big Internet Company \& 维生数工作室  Some rights reserved.
    \cc \hspace{-0.1in} Creative Commons BY-NC-SA
    \hfill
    \Huge \includegraphics[width=25pt]{./logo.pdf} \normalsize
    \hspace{5pt}http://big.vc
    \hfill
    \hspace{5pt}http://vtmer.com
    \vspace{8pt}
\end{beamercolorbox}%
}


\begin{document}
  \begin{frame}{}

    \noindent
    \begin{minipage}{\textwidth}
      \centering
      \includegraphics[width=0.95\textwidth]{./title-status-codes.pdf}
    \end{minipage}

    \begin{columns}
      \begin{column}{0.05\textwidth}
      \end{column}
      \begin{column}{0.9\textwidth}
        \begin{block}{}
          \large
            服务器对每个请求都会回复相应的 HTTP 状态代码。这些代码包括 3 为数字
            和一段文字描述(只起提示作用,而且可能会被翻译成其他语言)。这些状态
            代码根据一般的错误原因进行分类;同一个分类下的状态代码的最高位都相同。
            因此就算你不知道这个代码具体是什么,也能从最高位推断出它属于什么类别。
          \normalsize
        \end{block}
      \end{column}
      \begin{column}{0.05\textwidth}
      \end{column}
    \end{columns}

    \vspace{0.5in}

    \begin{block}{\huge 1XX 提示性}

      \vspace{0.3in}

      \rowcolors{3}{light-gray}{white}
      \begin{tabular}{p{0.3\textwidth} p{0.65\textwidth}}
        状态代码 & 描述\\ \hline
        100 Continue & 服务器已经接收到了请求头,客户端应该开始发送请求内容 \\
        101 Switching Protocols & 服务器已经收到切换协议的请求并且在进行切换 \\
        102 Processing (WebDAV) & 服务器已经收到请求。(在 WebDAV 中的长时间请求中使用以防止超时)\\
      \end{tabular}
    \end{block}
    \begin{block}{\huge 2XX 成功}

      \vspace{0.3in}

      \rowcolors{3}{light-gray}{white}
      \begin{tabular}{p{0.3\textwidth} p{0.65\textwidth}}
        状态代码 & 描述 \\ \hline
        200 OK & 对 HTTP 请求成功的标准回复 \\
        201 Created & 请求成功\emph{并且}(在服务器端)建立了新的资源 \\
        202 Accepted & 请求已经被接纳,但仍未处理完成。这个状态代码可能用在拒绝对已经开始处理的资源的请求上。\\
        203 Non-Authoritative Information & 请求已经被成功处理,但返回的内容可能来自未被信任的第三方 \\
        204 No Content & 请求已经被处理并且没有任何内容返回 \\
        206 Partial Content & 只有一部分的内容被送出(可用在重新开始被中断的下载中)\\
        207 Multi-Status (WebDAV) & 响应内容是 XML 文档并且可能对每个子请求包含不同的状态码 \\
        208 Already Reported (WebDAV) & 响应内容已经在之前的响应中出现过了并且不会再次发出 \\
      \end{tabular}
    \end{block}
    \begin{block}{\huge 3XX 重定向}

      \vspace{0.3in}

      \rowcolors{3}{light-gray}{white}
      \begin{tabular}{p{0.3\textwidth} p{0.65\textwidth}}
        状态代码 & 描述 \\ \hline
        300 Multiple Choices & 表明当前有多个目的地可供客户端前往 \\
        301 Moved Permanently & 请求的资源已经被永久地转移到其他地方了,以后的请求都应该转向给出的地址 \\
        302 Found & 已经找到资源(或者资源被临时转移到其他地方)。HTTP/1.0 中要求在该重定向中使用相同的请求方法;但实际上客户端会和 303 一样使用 GET 方法。见 303 和 307 \\
        303 See Other & 已经找到资源而且需要使用 GET 来访问。该代码在 HTTP/1.1 中加入以区分 302 含糊的行为。见 302 和 307 \\
        304 Not Modified & 资源和客户端上次缓存的一样 \\
        305 Use Proxy & 资源需要通过特定代理才能进行访问 \\
        307 Temporary Redirect & 需要对给定的地址使用相同的方法进行请求。该代码在 HTTP/1.1 中加入以区分 302 含糊的行为。见 302 和 303 \\
      \end{tabular}
    \end{block}
    \begin{block}{\huge 4XX 客户端错误}

      \vspace{0.3in}

      \rowcolors{3}{light-gray}{white}
      \begin{tabular}{p{0.3\textwidth} p{0.65\textwidth}}
        状态代码 & 描述 \\ \hline
        400 Bad Request & 请求中包含的语法错误导致请求不能被执行 \\
        401 Unauthorized & 客户端需要进行验证后才能访问资源 \\
        402 Payment Required & 这个状态代码计划用在微收费系统中,但是该系统的具体规范还没有制定出来;所以该代码很少用到 \\
        403 Forbidden & 客户端不允许访问该资源。通常是因为客户端通过验证但没有足够权限来进行访问 \\
        404 Not Found & 请求的资源没有找到,但有可能在未来出现 \\
        405 Method Not Allowed & 请求资源的方法不被支持 \\
        406 Not Acceptable & 服务器端不能以请求中的 ``Accept'' 头字段所指定的格式生成数据返回给客户端 \\
        407 Proxy Authentication Required & 客户端必须通过代理的验证 \\
        408 Request Timeout & 客户端没有在合理的时间里完成请求 \\
        409 Conflict & 请求因为资源状态的冲突而不能被完成 (例如尝试去更新一个已经在上次请求后被修改过的资源) \\
        410 Gone & 请求的资源已经永远地消失了,客户端不应该再去请求该资源 \\
        411 Length Required & 请求头中缺少访问资源所需要的 ``Content-Length'' 字段 \\
        412 Precondition Failed & 服务器不能满足客户端请求的前提条件 \\
        413 Request Entity Too Large & 请求主体大小超过服务器端可以处理的大小 \\
        414 Request-URI Too Long & 请求的资源地址太长 \\
        415 Unsupported Media Type & 请求中包含服务器端不能处理的媒体类型(MIME) \\
        416 Requested Range Not Satisfiable & 服务器端不能提供客户端请求的部分。(即有部分请求超出了所请求资源所包含的内容) \\
        417 Expectation Failed & 服务器端不能满足请求头中 ``Expect'' 字段的要求 \\
        418 I'm a teapot (HTCPCP) & 由实现了超文本咖啡壶控制协议的水壶所返回 \\
        420 Enhance Your Calm (Twitter) & 客户端的请求超出了配额(来自大麻文化中的俚语) \\
        422 Unprocessable Entity (WebDAV) & 由于语义上的错误导致请求不能被处理 \\
        423 Locked (WebDAV) & 资源当前被锁定 \\
        424 Failed Dependency (WebDAV) & 由于依赖的请求失败导致该请求失败 \\
        429 Too Many Requests & 客户端的请求超出了配额 \\
        431 Request Header Fields Too Large & 请求头中一个或全部字段的大小太大 \\
        444 No Response (Nginx) & 表明服务器端会关闭连接但不返回任何响应。(在 Nginx logs 中使用) \\
        449 Retry With (Microsoft) & 请求需要在执行某些操作后重新发起 \\
        450 Blocked by Windows Parental Controls (Microsoft) & 由于 Windows 的家长控制已经打开,导致对资源的请求被屏蔽 \\
        451 Unavailable For Legal Reasons (Internet Draft) & 打算用在描述被审查或被屏蔽的资源上。(来自小说 《华氏 451》)
      \end{tabular}
    \end{block}
    \begin{block}{\huge 5XX 服务器端错误}

      \vspace{0.3in}

      \rowcolors{3}{light-gray}{white}
      \begin{tabular}{p{0.3\textwidth} p{0.65\textwidth}}
        状态代码 & 描述 \\ \hline
        500 Internal Server Error & 当没有更好的状态代码适合的时候,用来表明服务器端发生错误 \\
        501 Not Implemented & 服务器端不能处理请求使用的方法 \\
        502 Bad Gateway & 作为网关或者代理的服务器端从上游获得了一个错误的响应(例如套接字被关闭) \\
        503 Service Unavailable & 资源暂时不能被访问,通常是因为该资源超载了或者需要下线维护 \\
        504 Gateway Timeout & 作为网关或者代理的服务器端由于上游超时而不能处理请求 \\
        505 HTTP Version Not Supported & 服务器端不支持请求使用的 HTTP 协议 \\
        507 Insufficient Storage (WebDAV) & 服务器端没有足够的储存空间来完成请求 \\
        508 Loop Detected (WebDAV) & 服务器端在处理请求的时候检测到死循环 \\
        509 Bandwidth Limit Exceeded & 习惯上用作说明带宽超出配额,但该状态代码并没有在任何一份 RFC 中出现过 \\
      \end{tabular}
    \end{block}
  \end{frame}
\end{document}
