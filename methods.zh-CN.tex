\documentclass[final,table]{beamer}

\usecolortheme{dove}

\usepackage{graphicx}
\usepackage{ctex}
\usepackage{fontspec, xunicode}
\usepackage[orientation=landscape, size=a1]{beamerposter}

\usepackage{scrextend}
\changefontsizes[20pt]{16pt}

\newcommand{\yes}{\color{green!60!black}{yes}}
\newcommand{\no}{\color{red}{no}}
\newcommand{\method}[1]{\textbf{#1}}
\definecolor{light-gray}{gray}{0.95}

\setlength{\tabcolsep}{18pt}
\renewcommand{\arraystretch}{1.1}
\renewcommand{\baselinestretch}{1.25}
\setlength{\parindent}{0em}

\usepackage{cclicenses}

\beamertemplatenavigationsymbolsempty

\setbeamertemplate{footline}
{%
\begin{beamercolorbox}{}
    \hspace{25pt} \normalsize Know Your HTTP \input{./version.tex} \copyright \hspace{0in} The Big Internet Company \& 维生数工作室  Some rights reserved.
    \cc \hspace{-0.1in} Creative Commons BY-NC-SA
    \hfill
    \Huge \includegraphics[width=25pt]{./logo.pdf} \normalsize
    \hspace{5pt}http://big.vc
    \hfill
    \hspace{5pt}http://vtmer.com
    \vspace{8pt}
\end{beamercolorbox}%
}


\begin{document}
  \begin{frame}{}
    \noindent
    \begin{minipage}{\textwidth}
      \centering
      \includegraphics[]{./title-methods.pdf}
    \end{minipage}
    
    \vspace{0.5in}

    \begin{columns}
      \begin{column}{0.1\textwidth}
      \end{column}
      \begin{column}{0.8\textwidth}
        \begin{block}{}
          \huge
          HTTP 请求方法,也叫请求动作,是用来说明对\emph{资源}(由统一资源标志符 URI 指向)
          使用的\emph{操作方法}。但实际上资源用什么方式来呈现以及被操作是由服务器端
          的程序来决定的。
          \normalsize
        \end{block}
      \end{column}
      \begin{column}{0.1\textwidth}
      \end{column}
    \end{columns}

    \vspace{0.5in}

    \begin{columns}
      \begin{column}{0.05\textwidth}
      \end{column}
      \begin{column}{0.9\textwidth}
        \begin{block}{}
          \Huge
          \rowcolors{3}{light-gray}{white}
          \begin{tabular}{r p{0.62\textwidth} l l l}
            请求方法 & 描述 & 安全? & 幂等? & 定义协议\\ \hline
            \method{GET} & 获取资源 & \yes & \yes & HTTP 1.0 \\
            \method{HEAD} & 获取资源,但只需要头部信息(资源的元信息) & \yes & \yes &HTTP 1.0 \\
            \method{POST} & 根据请求主体来处理资源 & \no & \no & HTTP 1.0 \\
            \method{PUT} & 根据请求主体的内容来创建或更新资源 & \no & \yes & HTTP 1.1 \\
            \method{DELETE} & 删除指定的资源 & \yes & \yes & HTTP 1.1 \\
            \method{OPTIONS} & 请求服务器端返回资源支持的请求方法 & \yes & \yes & HTTP 1.1 \\
            \method{TRACE} & 回显服务器收到的请求(主要用在测试或诊断) & \yes & \yes & HTTP 1.1 \\
            \method{CONNECT} & 将连接转换成管道方式(通常用在经由非加密的 HTTP 代理连接到 SSL 加密服务器端上) & --- & --- & HTTP 1.1 \\
            \method{PATCH} & 对资源进行部分修改 & \no & \yes & RFC-5789 \\
          \end{tabular}
        \end{block}

        \vspace{0.5in}

        \begin{columns}
          \begin{column}{0.3\textwidth}
            \begin{block}{\huge{安全的请求方法}}
              \Large
              那些被称作\emph{安全}的请求方法是指该请求不会对资源进行修改或者不会为
              服务器端整体的状态带来任何副作用。不安全的方法可能会产生副作用,例如:
              对资源进行修改,发出一封邮件或者初始化对信用卡的处理。
            \end{block}
          \end{column}
          \begin{column}{0.62\textwidth}
            \begin{block}{\huge{幂等的请求方法}}
              \Large
              一些请求方法是\emph{幂等}的,是指对指定的资源进行同样的操作任意多次
              都会产生同样的结果。例如 DELETE 请求就是幂等的,因为一个资源被删除之后,
              它将不能被再次删除。相反的, POST 请求是非幂等的,因为当你再次发出
              同样的请求后,服务器端可能会发出第二封相同的邮件或者对同一张信用卡
              再进行一次处理。

              由以上的定义我们可以知道,\emph{安全}的请求方法都是\emph{幂等}的。
            \end{block}
          \end{column}
        \end{columns}

      \end{column}
      \begin{column}{0.05\textwidth}
      \end{column}
    \end{columns}

  \end{frame}
\end{document}
